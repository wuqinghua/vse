%!TEX root = main.tex

\section{Related Work}
\label{sec:related}

There is a large and growing body of work~\cite{chen2012network,deshpande2010performance,sommers2012cell,yu2014can} characterizing the performance of mobile services in WiFi and cellular networks. Studies that are most closely related to ours have focused on analyzing how each network factor impacts the user-perceived performance in mobile services, like throughput and latency. Chen \etal~\cite{chen2012network} measured the network performance of mobile terminals in campus WiFi networks. They identified the dominating factors that affect network performance, and found that short flows in mobile terminals benefit from the large initial congestion window. Deshpande \etal~\cite{deshpande2010performance} measured cellular and WiFi performance in a metro area, via a laptop driven through the targeted area. Their results showed the higher availability yet lower performance of cellular networks, versus lower availability yet higher performance of WiFi networks. Sommers \etal~\cite{sommers2012cell} obtained similar results to \cite{deshpande2010performance}, through measurements with crowd-sourced data. Yu \etal~\cite{yu2014can} measured several popular mobile video telephony applications, and found that mobile video quality is highly vulnerable to bursty packet loss and high latency. 

There is also a growing literature~\cite{nikravesh2014mobile,deng2014wifi,huang2013depth,UM-CS-2012-022} on characterizing mobile network performance and usage. Nikravesh \etal~\cite{nikravesh2014mobile} inspected the cellular network performance, and found that the performance variations are partially induced by different carriers as well as spatial and temporal patterns. Huang \etal~\cite{huang2013depth} studied the impact of network protocol and application behaviors on performance in LTE. They found that TCP connections in LTE significantly under-utilize the available bandwidth, which is induced by both application behavior and TCP parameter setting. Chen \etal~\cite{UM-CS-2012-022} characterized the 3G/4G networks of US network carriers, in terms of throughput, packet loss, and latency. Moreover, they pointed out the benefits of multipath TCP when crossing congested network, for reliable and efficient TCP transfer.

The high packet loss rate and misbehavior of middleboxes in mobile networks may trigger costly timeout retransmission, leading to performance degradation for short flows. Jiang \etal~\cite{jiang2012tackling} studied bufferbloat and its impact on TCP performance in cellular networks. They revealed the severity of the bufferbloat problem and the induced long latency in current cellular networks. Wang \etal~\cite{Wang:2011:USM:2018436.2018479} studied the carriers' middlebox policies by conducting active measurement. They found that some middleboxes buffer disordered packets, which prevents the sender from receiving enough dupacks, causing timeout retransmissions. Flach \etal~\cite{flach2013reducing} evaluated the impact of packet loss in the flow tail on the performance in short flows. The results showed that packet loss at flow tail tend to trigger timeout retransmissions due to insufficient new data for transmission.

Compared to previous works, to the best of our knowledge, our paper is among the first to systematically analyze the impact of mobile network on the performance of both TCP upload and download in WiFi and cellular networks. We have quantitatively studied how each network factor, such as RTT, packet loss, reordering, timeout retransmission, impacts the TCP performance. We also have analyzed how each costly timeout retransmission is triggered and discussed possible solutions.
