%!TEX root = main.tex

\section{Related Work}
\label{sec:related}

There is a large and growing body of work~\cite{chen2012network,deshpande2010performance,sommers2012cell,sharma2010goodput,yu2014can} characterizing the performance of mobile services in WiFi and cellular networks. Studies that are most closely related to ours have been focusing on analyzing how each network factor impacts the user-perceived performance in mobile services, like throughput and latency. Chen \etal~\cite{chen2012network} measured the network performance of mobile terminals in campus WiFi network. They identified the dominate factors that affect network performance, and found that short flows in mobile terminals benefit from the large initial congestion window. Deshpande \etal~\cite{deshpande2010performance} measured cellular and WiFi performance in a metro area, via a laptop driven through the targeted area. Their results showed the higher availability yet lower performance of cellular network, versus lower availability yet higher performance of WiFi network. Sommers \etal~\cite{sommers2012cell} obtained similar results as in \cite{deshpande2010performance}, through measurement with crowd-sourced data. Yu \etal~\cite{yu2014can} measured several popular mobile video telephony applications, and found that mobile video quality is highly vulnerable to bursty packet loss and high latency. 

There is also a growing literature~\cite{nikravesh2014mobile,deng2014wifi,huang2013depth,UM-CS-2012-022} on characterizing mobile network performance and usages. Nikravesh \etal~\cite{nikravesh2014mobile} inspected the cellular network performance, and found that the performance variation are partially induced by different carriers as well as spatial and temporal patterns. Huang \etal~\cite{huang2013depth} studied the impact of network protocol and application behaviors on performance in LTE network. They found that TCP connections in LTE network significantly under-utilize the available bandwidth, which is induced by both application behavior and TCP parameter setting. Chen \etal~\cite{UM-CS-2012-022} characterized the 3G/4G networks of US network carriers, in terms of throughput, packet loss, and latency. Moreover, they pointed out the benefits of multipath TCP when passing congested network, for reliable and efficient TCP transfer.

The high packet loss rate and misbehaviors of middleboxes in mobile network may trigger costly timeout retransmission, which is great performance degradation for short flow. Jiang \etal~\cite{jiang2012tackling} studied the bufferbloat and its impact on TCP performance in cellular networks. They revealed the severity of bufferbloat problem and the induced long latency in current cellular networks. Wang \etal~\cite{Wang:2011:USM:2018436.2018479} studied the carriers' middlebox policies by conducting active measurement. They found that some middleboxes buffer disordered packets, which prevents sender to receive sufficient dupacks, which would cause timeout retransmission. Flach \etal~\cite{flach2013reducing} evaluated the impact of packet loss at flow tail on the performance in short flows. The results showed that packet loss at flow tail might trigger timeout retransmission due to insufficient new data for transmission.

As opposed to previous research, to the best of our knowledge, our paper is among the first to systematically analyze the impact of mobile network on the performance of both TCP uploading and downloading flows in WiFi and cellular networks. We quantitatively studied how each network factor, like RTT, packet loss, reordering, timeout retransmission, impacts the TCP performance. We also analyzed how each costly timeout retransmission is triggered, and proposed possible solutions to eliminate them.