%!TEX root = main.tex

\section{Conclusion}
\label{sec:conclude}

Voice search service consists of voice recognition and web search in two successive flows, in which one is uploading voice data and the other is downloading web search result. The complexity and uniqueness of voice search motivate us to understand the performance and its causes. In this paper, we first defined finish time in both voice recognition and web search to depict the performance related to network, and broke the analysis into two phases. By using a unique dataset collected from a voice search service, we systematically analyzed the impact of network factors on finish time in both phases. We have made several key observations and discussed their implication on the design of voice search system to reduce finish time. We observed that SYN retransmission occupies a non-negligible fraction of WiFi flows, in both voice recognition and web search. We also found that when there is packet loss, both 3G and WiFi flows tend to suffer from timeout retransmission, due to insufficient data packets for transmission. We further classified the timeout retransmissions and found the different compositions of timeout retransmissions between WiFi flows and 3G flows. These observations constitute valuable insights for improving the user-perceived performance in voice search service.