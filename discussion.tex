%!TEX root = main.tex

\section{Summary and Discussion}
\label{sec:discuss}

In this section, we first summarise the findings from our analysis of voice recognition (Section~\ref{sec:web_search}) and web search (Section~\ref{sec:web_search}), and follow up with a discussion of their implications for the state-of-the-art and future work.

Let us first summarize our main findings:

\begin{itemize}
	\item Flows in both voice recognition and web search are short. Ideally, voice recognition flows will complete in 2 RTTs (one for 3WHS, one for voice data upload), and web search flows will complete in no more than 5 RTTs (one for 3WHS, 4 for web search results). However, we see that more than 10\% of voice recognition flows take 0.5 second to complete, and 20\% of web search flows complete in 1 second.

	\item Compared to flows in 3G, WiFi flows experience longer completion time, both for voice recognition and web search. This result unexpected in light of previous work~\cite{deshpande2010performance,sommers2012cell}. The issues with WiFi flows are induced by both higher packet loss rate (3\% in WiFi versus 0.9\% in 3G) and larger RTTs.

	\item WiFi flows suffer from less packet reordering in voice recognition, and more packet loss in web search than 3G flows. However, due to larger RTTs, senders in WiFi need more time to recover from these events.

	\item In both voice recognition and web search, if there is no packet loss, the RTT is the key parameter for the flow completion time. When there is packet loss, WiFi flows are more likely to suffer from timeout retransmission. In this situation, timeout retransmission, requiring tens of or even hundreds of RTTs for recovery, will dominate the flow completion time.

	\item Timeout retransmission can happen anywhere between the SYN and the last few data packets. In fact, tail retransmission contributes most of the timeout retransmissions in web search, especially for 3G flows. Double retransmission and packet delay retransmission on the other hand play a more important role for WiFi flows.
\end{itemize}

The above findings can help service providers optimize the transmission performance for the voice search service. First, as smaller RTT leads to shorter finish time. For this, service providers can deploy front-end servers nearer to clients or ensure the lowest latency server is selected. Second, mobile applications can keep persistent TCP connections, which will mitigate the impact of SYN retransmission on user-perceived performance. Third, smarter strategies in the dynamic or joint selection of 3G and WiFi technologies (\eg, through multipath TCP) could be leveraged for more reliable service~\cite{UM-CS-2012-022,Chen:2013:MSM:2504730.2504751}, enabling dynamically offloading traffic from congested network segments without breaking existing connections. Fourth, service providers can deploy solutions such as TLP~\cite{flach2013reducing} to reduce the number of tail retransmissions in short flows. However, other types of timeout retransmission require significant effort from service providers for mitigation, such as redesigning TCP retransmission mechanisms to eliminate double retransmission, and cooperating with network providers to debug misbehaving middleboxes. Last but no least, due to TCP's sender driven nature, voice search service providers should adopt different strategies for performance optimizations in each phase of the TCP flow lifetime. For example, voice search applications would not modify the TCP stack, but would maintain a pool of TCP connections and take advantage of multipath TCP to reduce the flow completion time. In contrast, a web search server, being the sender in the connection, would modify the TCP mechanism to handle packet loss more efficiently and transmit more data packets per time unit.

Our observations indicate that WiFi provides worse quality of service than 3G for mobile voice search, unlike previous studies~\cite{sommers2012cell}. We conjecture that the most likely reason is that the WiFi networks observed in this paper have a much higher packet loss rate than the 3G networks (3.0\% versus 0.9\%). We have shown that packet loss in WiFi heavily impacts performance. In fact, it has been reported in the past that while China's cellular network has a comparable packet loss rate with other western countries' networks, its fixed Internet has a much higher packet loss rate~\cite{HeikkinenB12}.

Unfortunately, our data has limited visibility on the client-side, so we are limited in our understanding of its impact on TCP performance. For example, this prevents us from inferring the exact packet loss in voice recognition flows. Second, as the flows belonging to each phase are collected separately at the server-side, we could not match each \emph{voice recognition}-\emph{web search} flow pair. This limits our quantitative analysis of user perceived latency for individual voice search sessions. Third, our dataset contain a limited number of 4G sessions due to the low coverage of 4G network in the considered region. The deployment of 4G however is improving in China, which might enable us to have a look at voice search performance in 4G network in the future.

%We believe that we could obtain more findings if the limitation of dataset are solved, which are left as future work.
