%!TEX root = main.tex

\section{Summary and Discussion}
\label{sec:discuss}

Our study mainly focuses on the TCP performance of voice search service at server side. As voice search session consists of voice recognition and web search phases, we inspect the main factors that may impact the finish time of flows corresponding to each phase. After quantitative measurement of the flows, we obtain the following findings.

\begin{itemize}
	\item Flows in both voice recognition and web search are short flows. Ideally voice recognition flows will complete in 2 RTT, and web search flows will complete in no more than 5 RTT's. However, we could see outliers in the dataset that more than 10\% of voice recognition flows complete in 0.5 second, and 20\% of web search flows complete in 1 second.

	\item Compared to flows in 3G network, WiFi flows experience longer finish time in both phases. This result is opposite to that in previous studies~\cite{deshpande2010performance,sommers2012cell}. The deficiency of WiFi flows are induced by both longer RTT and higher packet loss rate.

	\item WiFi flows suffer from less packet reordering in voice recognition, and more packet loss in web search than 3G flows. However, due to longer RTT, senders in WiFi network need more time to handle these congestion events.

	\item In both voice recognition and web search, if there is no packet loss, RTT is the key parameter in the finish time. When there is packet loss, WiFi flows are more likely to suffer from timeout retransmission. In this situation, timeout retransmission, occupying tens of or even hundreds of RTT's for recovery, will dominate the finish time.
\end{itemize}

The above findings could better guide service providers to optimize the transmission performance in voice search service. First, as smaller RTT leads to shorter finish time, service provider could deploy front-end servers nearer to clients to achieve shorter latency. Second, mobile applications could maintain TCP connections before acquiring services, which will mitigate the impact of SYN retransmission on user-perceived performance. Third, the 3G network access and WiFi network access could be combined under multipath TCP for more reliable service access~\cite{UM-CS-2012-022}, which enables dynamically offloading traffic from congested network, without breaking existing connections. Forth, service provider could deploy solutions like TLP~\cite{flach2013reducing} to reduce the number of tail retransmission in short flows. However, other types of timeout retransmission require great effort from service provider for mitigation, such as redesigning TCP retransmission mechanism to eliminate double retransmission, and cooperating with network provider to debug misbehaving middleboxes. Last but no least, due to TCP's sender driven property, voice search service provider should adopt different strategies for performance optimization in each phase. For example, voice search application could not modify the TCP stack, but it can maintain connection pool, and use multipath TCP to reduce finish time. In contrast, web search server, as the sender in the connection, could modify the TCP mechanism to handle packet loss more efficiently and transmit more data packets in unit time.

The limitation of our data collection also restricts our understanding of the impact of TCP performance in voice search service. First, our dataset are collected at server sides, thus we could not determine the exact packet loss information in voice recognition flow. As packet loss plays an important role in TCP performance, which also demonstrates the difficulty of performance analysis of TCP uploading flows at server side. Second, as the flows belonging to each phase are collected separately, we could not match each \emph{voice recognition}-\emph{web search} pair. Thus we are not capable of quantitatively understanding how each phase impacts the user-perceived performance in voice search, and how each network factor impact different in the two phases. Third, from our dataset, we find that RTT in WiFi network is about 2-3 times larger than that in 3G network, which is contradictory to previous studies like \cite{sommers2012cell}. This difference might be induced by the unique network characteristics in the region network where we collected the dataset. However, we could verify our findings through more data collection from multiple region networks.

We believe that we could obtain more findings if the limitation of dataset are solved, which are left as future work.