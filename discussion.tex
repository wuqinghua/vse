%!TEX root = main.tex

\section{Summary and Discussion}
\label{sec:discuss}

Our study mainly focuses on the TCP performance of voice search service at server side. As voice search session consists of voice recognition and web search phases, we inspect the main factors that may impact the finish time of flows corresponding to each phase. To summarize, we obtain the following findings.

\begin{itemize}
	\item Flows in both voice recognition and web search are short flows. Ideally voice recognition flows will complete in 2 RTTs (one for 3WHS, one for voice data uploading), and web search flows will complete in no more than 5 RTTs (one for 3WHS, 4 for web search results). However, we could see outliers in the dataset that more than 10\% of voice recognition flows complete in 0.5 second, and 20\% of web search flows complete in 1 second.

	\item Compared to flows in 3G network, WiFi flows experience longer finish time in both phases. This result is opposite to that in previous studies~\cite{deshpande2010performance,sommers2012cell}. The deficiency of WiFi flows are induced by both higher packet loss rate (3\% in WiFi versus 0.9\% in 3G) and longer RTT.

	\item WiFi flows suffer from less packet reordering in voice recognition, and more packet loss in web search than 3G flows. However, due to longer RTT, senders in WiFi network need more time to handle these congestion events.

	\item In both voice recognition and web search, if there is no packet loss, RTT is the key parameter in the finish time. When there is packet loss, WiFi flows are more likely to suffer from timeout retransmission. In this situation, timeout retransmission, requiring tens of or even hundreds of RTTs for recovery, will dominate the finish time.

	\item Timeout retransmission can happen as early as for SYN and as late as for the last a few data packets. In fact, tail retransmission contributes most of the timeout retransmissions in web search, especially for 3G flows. Double retransmission and packet delay retransmission on the other hand play a more important role for WiFi flows.
\end{itemize}

The above findings can better guide service providers to optimize the transmission performance in voice search service. First, as smaller RTT leads to shorter finish time, service provider can deploy front-end servers nearer to clients to achieve shorter latency. Second, mobile applications can maintain TCP connections before acquiring services, which will mitigate the impact of SYN retransmission on user-perceived performance. Third, the 3G network access and WiFi network access can be used together through multipath TCP for more reliable service access~\cite{UM-CS-2012-022,Chen:2013:MSM:2504730.2504751}, which enables dynamically offloading traffic from congested network, without breaking existing connections. Forth, service provider can deploy solutions like TLP~\cite{flach2013reducing} to reduce the number of tail retransmissions in short flows. However, other types of timeout retransmission require great effort from service provider for mitigation, such as redesigning TCP retransmission mechanism to eliminate double retransmission, and cooperating with network provider to debug misbehaving middleboxes. Last but no least, due to TCP's sender driven property, voice search service provider should adopt different strategies for performance optimization in each phase. For example, voice search application could not modify the TCP stack, but it can maintain connection pool, and use multipath TCP to reduce finish time. In contrast, web search server, as the sender in the connection, could modify the TCP mechanism to handle packet loss more efficiently and transmit more data packets in unit time.

Our observations indicate that WiFi network provides better quality of service than 3G network for mobile voice search, which is contradictory to previous studies like \cite{sommers2012cell}. The most likely reason is that the WiFi network that we considered in this paper has a much higher packet loss rate than 3G network (3.0\% versus 0.9\%). We have shown that packet loss in WiFi network heavily hurt the performance. In fact, it has been found that while China's cellular network has a comparable packet loss rate with other western countries' networks, its fixed Internet has a much higher packet loss rate than others \cite{HeikkinenB12}.

The limitation of our data collection also restricts our understanding of the impact of TCP performance in voice search service. First, our datasets are collected at server sides, thus we are unable to determine the exact packet loss information in voice recognition flows. This demonstrates the difficulty of performance analysis of TCP uploading flows at server side. Second, as the flows belonging to each phase are collected separately at server side, we could not match each \emph{voice recognition}-\emph{web search} pair. This limits the quantitatively understanding of user perceived latency for individual voice search sessions. Third, our datasets contain very limited number of 4G sessions due to the low coverage of 4G network in the region we considered. The deployment of 4G however is becoming larger, which might enable us to have a look at voice search performance in 4G network in near future.

%We believe that we could obtain more findings if the limitation of dataset are solved, which are left as future work.
