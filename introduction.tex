%!TEX root = main.tex

\section{Introduction}
\label{sec:intro}

Voice search service is becoming gradually more popular due to its convenience on mobile devices in the past years. A recent report showed that more than half of teens use voice search more than once a day~\cite{voice_search_report}. A voice search session consists of two successive phases: voice recognition and web search. First, mobile terminal transmits the speech data to voice recognition server and gets the recognized keyword. After that, mobile terminal queries the keyword and obtains the search results. In deed, the two phases are carried out in two independent flows destined to different servers. One of the flows is uploading data, the other is downloading data, and both flows are short (less than 100 packets).

We are motivated by the fact that the long latency and high packet loss rate in mobile and wireless network may greatly degrade user-perceived performance in voice search service. There is already a large body of work \cite{sommers2012cell,yu2014can,chen2012network,sharma2010goodput} studying the impact of network quality on user perceived performance of mobile applications. However, most of them focus on the downloading efficiency of long flows in wireless and mobile network, which is not applicable to voice search service, which contains both uploading and downloading short flows. The complexity and uniqueness of voice search service pose great challenges in understanding the user-perceived performance and its causes in voice search service.

To this end, we sampled a unique dataset from voice search servers in one of the top 3 search service providers in China. The dataset spans over two weeks in April 2015, consisting of about 1 million voice recognition flows and 3 million web search flows, in pcap format. To investigate the impact of network quality on the user-perceived performance of voice search service, we break the analysis into two independent segments: voice recognition and web search. In each phase, we depict the performance of data transmission as finish time, which is defined as the duration from that sender establishes connection, to that the last byte is acknowledged. In the finish time, the time consumed unrelated to data transmission is omitted, like translating speech into text. In voice recognition phase, server does not pose sufficient parameters for analysis, as TCP is a sender-driven protocol. To tackle this problem, we use heuristics to estimate RTT and timeout retransmission to investigate their impact on finish time. In web search phase, we partition the web search phase into three stages: 3-way handshake, slow start, and congestion avoidance. We examine the impact of the three stages on finish time, and inspect the key parameter in each stage. Our main findings and observations are summarized as follows.

\begin{itemize}
\item We observe that mobile terminals in WiFi and 3G networks have different daily usage patterns, 3G users tend to have more voice search requests at morning before work, and WiFi users tend to request voice search service more at evening after work. Besides, web search flows in WiFi network have smaller size than those in 3G network. These differences may be induced by that requests in WiFi network are more likely from tablets, and requests in 3G network are more likely from phones. 
\item When there are seldom packet loss events, RTT is the key parameter in the performance of both voice recognition and web search flows, as smaller RTT value enables sender to obtain acknowledgments in shorter time. However, when there are packet losses, sender needs time to recover the lost packets, thus the time of loss recovery will dominate the performance.
\item 3-way handshake occupies a large fraction of finish time in both voice recognition and web search flows. Moreover, 2.6\% of voice recognition flows and 4.4\% of web search flows in WiFi network suffer from SYN retransmission, which makes the time for establishing connections longer than 1 second. 
\item Compared with flows in 3G network, WiFi flows experience much longer finish time, due to higher packet loss rate. Even under the same number of lost packets, WiFi flows are more likely to suffer from timeout retransmission.
\item We further classify timeout retransmissions during data transfer into different groups by how they are triggered. In our dataset, tail retransmission, double retransmission, and packet delay retransmission dominate all the timeout retransmissions. Tail retransmission occupies the largest fraction in both WiFi and 3G flows, as it is more likely to occur in short flows. Compared with 3G flows, WiFi flows have more double retransmissions and packet delay retransmissions, due to high packet loss rate and misbehaving middleboxes in WiFi network.
\end{itemize}

Our findings highlight the impact of RTT, packet loss, and timeout retransmission on the finish time of flows. We believe that these findings provides valuable insights on improving the performance of voice search service, including content distribution, TCP protocol optimization, application design, and middlebox diagnosis.

The remaining of this paper is organized as follows. Section~\ref{sec:dataset} describes the dataset. Section~\ref{sec:voice} and Section~\ref{sec:web_search} carry out an in-depth analysis on the performance of voice recognition and web search separately, and inspect the performance causes. Section~\ref{sec:discuss} discusses the implication of the observations and the possible shortcomings of this work. Section~\ref{sec:related} compares the related work and Section~\ref{sec:conclude} concludes our work. 