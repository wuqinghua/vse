%!TEX root = main.tex

\section{Introduction}
\label{sec:intro}

Voice search service has become gradually more popular over the last few years due to its convenience on mobile devices. A recent report showed that more than half of teens use voice search more than once a day~\cite{voice_search_report}. A voice search session consists of two successive phases: voice recognition and web search. First, the mobile terminal transmits the speech data to a voice recognition server and gets the recognized keyword(s). After that, the mobile terminal queries the keyword(s) and obtains the search results. The two phases are carried out across two independent flows, established to different servers. One of the flows is uploading data (voice search), the other is downloading data (web search result), and both flows are short (less than 100 packets).

In this paper, we are motivated by the fact that the latency and loss rates in mobile and wireless network are able to significantly degrade user-perceived performance in such voice search service. There is already a large body of work \cite{sommers2012cell,yu2014can,chen2012network,sharma2010goodput} studying the impact of network quality on user perceived performance of mobile applications. However, most of them focus on the downloading efficiency of relatively long flows in wireless and mobile network, which is not applicable to the voice search service, which contains mostly short flows, both in the upload and download directions. The complexity and uniqueness of voice search is a great opportunity to better understand the user-perceived performance in this type of traffic.

To this end, we obtained a unique dataset from voice search servers in one of the top 3 search service providers in China. The dataset spans over two weeks in April 2015, consisting of about 1 million voice recognition flows and 3 million web search flows, in the usual pcap format. To investigate the impact of the network on the user-perceived performance in voice search flows, we break down the analysis into two independent segments: voice recognition (Section~\ref{sec:voice}) and web search (Section~\ref{sec:web_search}). In each part, we analyse the performance of data transmission through the \textit{finish time}, defined as the duration between the time of the connection establishment by the sender and the time the last byte is acknowledged. For the voice recognition phase, we omit the time consumed unrelated to data transmission, such as translating speech into text, as it is not directly related to the network. As we observe flows from the server-side, we do not have visibility of when the clients sent their TCP segments. Therefore, we propose heuristics to estimate the RTT and timeout retransmission, and investigate their impact on the finish time. For the web search phase, we partition the web search phase into three stages: 3-way handshake, slow start, and congestion avoidance. We examine the impact of the three stages on the finish time, and inspect the key parameter in each stage. Our main findings and observations are the following:

\begin{itemize}
\item We observe that mobile terminals using WiFi and 3G networks have different daily usage patterns: 3G users tend to have more voice search requests during the morning before work, while WiFi users tend to to make voice search requests during the evening after work. Web search flows in WiFi are smaller than those in 3G. These differences may be induced by the dominant type of device for each access technology: tablets in WiFi and smartphones in 3G.
\item For flows that do not suffer from packet losses, the RTT is the key parameter in the performance of both voice recognition and web search, as smaller RTT values enable the sender to obtain acks faster. However, when there are packet losses, the sender needs time to recover, and the time taken for loss recovery will dominate the performance.
\item The 3-way handshake occupies a large fraction of the finish time in both voice recognition and web search flows. Moreover, 2.6\% of voice recognition flows and 4.4\% of web search flows in WiFi suffer from SYN retransmission, making the time for establishing connections larger than 1 second.
\item Compared to flows in 3G, WiFi flows experience a much longer finish time, due to higher packet loss rates. Even under the same number of lost packets, WiFi flows are more likely to suffer from timeout retransmissions.
\item We further classify timeout retransmissions during data transfer into different groups by how they are triggered. In our dataset, tail retransmissions, double retransmissions, and packet delay retransmissions dominate timeout retransmissions. Tail retransmissions occupy the largest fraction in both WiFi and 3G flow time, as it is more likely to occur in short flows. Compared to 3G flows, WiFi flows have more than twice the number of retransmissions and packet delay retransmissions, due to high packet loss rates as well as the presence of middleboxes buffering packets.
\end{itemize}

Our findings highlight the impact of RTT, packet loss, and timeout retransmissions on the flow finish time in this peculiar but increasingly popular service. We believe that these findings provide valuable insight into the performance of voice search service, and the importance of TCP protocol optimizations, application design, and middlebox diagnosis, and their interactions.

The remaining of this paper is organized as follows. Section~\ref{sec:dataset} describes the dataset. Section~\ref{sec:voice} and Section~\ref{sec:web_search} carry out an in-depth analysis on the performance of voice recognition and web search separately, and inspect the performance causes. Section~\ref{sec:discuss} discusses the implication of the observations and the possible shortcomings of this work. Section~\ref{sec:related} compares the related work and Section~\ref{sec:conclude} concludes our work. 