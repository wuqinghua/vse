\documentclass{sig-alternate}

\usepackage{url}
%\usepackage[caption=false]{subfig}
\usepackage{subcaption}

%\usepackage{amssymb}
%\usepackage{amsmath}
\usepackage{color}
\usepackage{algorithm}
\usepackage{algpseudocode}
\usepackage{booktabs}
\usepackage{array}
\usepackage{xspace}
\usepackage{graphicx}

% shortcut
\newcommand{\etc}{{etc.}\@\xspace}
\newcommand{\ie}{{i.e.,}\@\xspace}
\newcommand{\etal}{{et al.}\@\xspace}
\newcommand{\eg}{{e.g.,}\@\xspace}
\newcommand{\aka}{{aka}\@\xspace}
\newcommand{\comment}[1] {\textbf{\color{red}{Comment: #1}}}
\renewcommand{\algorithmicforall}{\textbf{for each}}

\newtheorem{theorem}{Theorem}
\newtheorem{definition}{Definition}

\usepackage{multirow}
\newcolumntype{L}[1]{>{\raggedright\let\newline\\\arraybackslash\hspace{0pt}}m{#1}}
\newcolumntype{C}[1]{>{\centering\let\newline\\\arraybackslash\hspace{0pt}}m{#1}}
\newcolumntype{R}[1]{>{\raggedleft\let\newline\\\arraybackslash\hspace{0pt}}m{#1}}


% \usepackage{setspace} % squeeze the vertical space in algorithm

% \makeatletter
% \renewcommand{\ALG@beginalgorithmic}{\small}
% \makeatother

\graphicspath{{figs/}}

\begin{document}

\title{A Server-side View of TCP Performance \\of Mobile Voice Search}

\numberofauthors{1}
\author{
	\alignauthor Qinghua Wu$^\dag$ $^\star$, Zhenyu Li$^\dag$, Qian Deng$^\dag$, Jianer Zhou$^\dag$ $^\star$, Steve Uhlig$^\ddag$, \\ Peter Steenkiste$^\sharp$, Gaogang Xie$^\dag$ \\
    \affaddr{\vspace{0.5em} 
	$^\dag$Institute of Computing Technology, Chinese Academy of Sciences, China\\ 
	$^\star$University of Chinese Academy of Sciences, China\\
	$^\ddag$Queen Mary, University of London\\
	$^\sharp$Carnegie Mellon University}\\
    \email{\vspace{0.5em} \{wuqinghua, zyli, dengqian, zhoujianer, xie\}@ict.ac.cn, steve@eecs.qmul.ac.uk, prs@cs.cmu.edu}
}

\maketitle

\begin{abstract}
Voice search service is becoming gradually more popular due to its convenience on mobile devices in the past years. A voice search session consists of two successive phases: voice recognition and web search. We collect a unique dataset from voice search servers, and bread the analysis into two segments. In voice recognition phase, we investigate the impact of RTT, packet reordering, and timeout retransmission on finish time. In web search phase, we partition the web search phase into three stages, examine the impact of the three stages on finish time, and inspect the key parameter in each stage. We obtain the following observations based on the analysis above. Despite the small flow size, there are outlier flows experiencing extremely long finish time. When there is no packet loss, RTT is the key parameter of finish time. When there is packet loss, compared with 3G flows, WiFi flows are more likely to suffer from timeout retransmission, which will dominate the finish time. After classification of these timeout retransmissions, we find that tail retransmission occupies the largest fraction in both WiFi flows (38.3\%) and 3G flows (69.6\%). However, WiFi flows also suffer from double retransmission (33.8\%) and packet delay retransmission (27.4\%). We believe these findings provide valuable insight on improving the performance of voice search service.
\end{abstract}

%!TEX root = main.tex

\section{Introduction}
\label{sec:intro}

Voice search service has become gradually more popular over the last few years due to its convenience on mobile devices. A recent report showed that more than half of teens use voice search more than once a day~\cite{voice_search_report}. A voice search session consists of two successive phases: voice recognition and web search. First, the mobile terminal transmits the speech data to a voice recognition server and gets the recognized keyword(s). After that, the mobile terminal queries the keyword(s) and obtains the search results. The two phases are carried out across two independent flows, established to different servers. One of the flows is uploading data (voice search), the other is downloading data (web search result), and both flows are short (less than 100 packets).

In this paper, we are motivated by the fact that the latency and loss rates in mobile and wireless network are able to significantly degrade user-perceived performance in such voice search service. There is already a large body of work \cite{sommers2012cell,yu2014can,chen2012network,sharma2010goodput} studying the impact of network quality on user perceived performance of mobile applications. However, most of them focus on the downloading efficiency of relatively long flows in wireless and mobile network, which is not applicable to the voice search service, which contains mostly short flows, both in the upload and download directions. The complexity and uniqueness of voice search is a great opportunity to better understand the user-perceived performance in this type of traffic.

To this end, we obtained a unique dataset from voice search servers in one of the top 3 search service providers in China. The dataset spans over two weeks in April 2015, consisting of about 1 million voice recognition flows and 3 million web search flows, in the usual pcap format. To investigate the impact of the network on the user-perceived performance in voice search flows, we break down the analysis into two independent segments: voice recognition (Section~\ref{sec:voice}) and web search (Section~\ref{sec:web_search}). In each part, we analyse the performance of data transmission through the \textit{finish time}, defined as the duration between the time of the connection establishment by the sender and the time the last byte is acknowledged. For the voice recognition phase, we omit the time consumed unrelated to data transmission, such as translating speech into text, as it is not directly related to the network. As we observe flows from the server-side, we do not have visibility of when the clients sent their TCP segments. Therefore, we propose heuristics to estimate the RTT and timeout retransmission, and investigate their impact on the finish time. For the web search phase, we partition the web search phase into three stages: 3-way handshake, slow start, and congestion avoidance. We examine the impact of the three stages on the finish time, and inspect the key parameter in each stage. Our main findings and observations are the following:

\begin{itemize}
\item We observe that mobile terminals using WiFi and 3G networks have different daily usage patterns: 3G users tend to have more voice search requests during the morning before work, while WiFi users tend to to make voice search requests during the evening after work. Web search flows in WiFi are smaller than those in 3G. These differences may be induced by the dominant type of device for each access technology: tablets in WiFi and smartphones in 3G.
\item For flows that do not suffer from packet losses, the RTT is the key parameter in the performance of both voice recognition and web search, as smaller RTT values enable the sender to obtain acks faster. However, when there are packet losses, the sender needs time to recover, and the time taken for loss recovery will dominate the performance.
\item The 3-way handshake occupies a large fraction of the finish time in both voice recognition and web search flows. Moreover, 2.6\% of voice recognition flows and 4.4\% of web search flows in WiFi suffer from SYN retransmission, making the time for establishing connections larger than 1 second.
\item Compared to flows in 3G, WiFi flows experience a much longer finish time, due to higher packet loss rates. Even under the same number of lost packets, WiFi flows are more likely to suffer from timeout retransmissions.
\item We further classify timeout retransmissions during data transfer into different groups by how they are triggered. In our dataset, tail retransmissions, double retransmissions, and packet delay retransmissions dominate timeout retransmissions. Tail retransmissions occupy the largest fraction in both WiFi and 3G flow time, as it is more likely to occur in short flows. Compared to 3G flows, WiFi flows have more than twice the number of retransmissions and packet delay retransmissions, due to high packet loss rates as well as the presence of middleboxes buffering packets.
\end{itemize}

Our findings highlight the impact of RTT, packet loss, and timeout retransmissions on the flow finish time in this peculiar but increasingly popular service. We believe that these findings provide valuable insight into the performance of voice search service, and the importance of TCP protocol optimizations, application design, and middlebox diagnosis, and their interactions.

The remaining of this paper is organized as follows. Section~\ref{sec:dataset} describes the dataset. Section~\ref{sec:voice} and Section~\ref{sec:web_search} carry out an in-depth analysis on the performance of voice recognition and web search separately, and inspect the performance causes. Section~\ref{sec:discuss} discusses the implication of the observations and the possible shortcomings of this work. Section~\ref{sec:related} compares the related work and Section~\ref{sec:conclude} concludes our work. 
%!TEX root = main.tex

\section{Data}
\label{sec:dataset}

We begin this section with the description of our packet-level data from a mobile voice search service, and then provide high-level statistics of TCP performance. 

\subsection{Data Collection}

\begin{figure}[th]
	\centering
	\includegraphics[width=0.8\linewidth]{voice_search_process}
	\caption{Voice search engine infrastructure.}
	\label{fig:voice_search}
\end{figure}

We collect mobile voice search data from one of the top 3 search service providers in China. This service provider serves more than 400 million users per day. A voice search initialized from a mobile terminal consists of two successive phases as shown in Figure~\ref{fig:voice_search}: \emph{voice recognition} (\ie recognize speech to query text) and \emph{web search} (\ie search with returned query text). Voice recognition and web search are served by different servers through HTTP. As such, a voice search session consists of two separate TCP flows.

We collect packet-level traces from front-end servers\footnote{Note that our traces are from a subset (3) of all servers, and given the load-balancing done across these, our data likely represents a uniform random flow sample.} of both voice recognition and web search, resulting in two datasets that correspond to the two phases of voice search. The front-end servers from which we obtained the data provide services for mobile users in the same geographical locations. As the service provider relies on geographically-biased server selection, we assume that the flows in the two datasets cross networks with similar characteristics. The web search servers provide search services for both voice search and traditional user-type search. The datasets were collected in April 2015 for two weeks. In total, we obtained about 1 million voice recognition flows and 3 million web search flows.

%We report measurement results for each of the two phases of voice search.

All voice search requests are originated from mobile apps, especially from the Android platform, either via cellular or WiFi access network. About 2.5\% of the voice recognition flows and 6.7\% of the web search flows originate from the cellular network, made of a mixture of technologies: 2.5G, 3G and 4G\footnote{The cellular network type was inferred using the HTTP header field ``x-up-bear-type''.}. However, we observed a very limited number of 2.5G and 4G flows (less than 0.5\% of the total flows) in our data\footnote{Indeed, 4G is still in its initial stages of deployment in China and has limited coverage.} and thus omit them from our analysis. In total, they make up about 2.5\% of the flows.

\begin{figure}[th]
\centering
\includegraphics[width=0.8\linewidth]{voice_time_rate}
\caption{Distribution of voice search requests in a day.}
\label{fig:voice_time_rate}
\end{figure}

Figure~\ref{fig:voice_time_rate} shows the distribution of voice search flows over time of a day. Flows are binned into 1-hour frame and we report the ratio of flows in each interval divided by the daily total. Not surprisingly, we observe a sharp increase of the search volume from 5AM to 10AM. The search volume from 3G remains relatively stable over the day and declines around 10PM, while we observe continuous growth of the WiFi search which reaches its peak at 7PM. The daily trend for web search flows is similar (not show). The daily search volume variation trend we observe is very similar to the one previously reported by \cite{Song:2013:EEU:2488388.2488493} for Bing mobile search.

\subsection{TCP Performance Measures}

\begin{figure}[ht]
\centering
\begin{subfigure}[b]{0.6\linewidth}
	\includegraphics[width=\textwidth]{voice_estimate_rtt}
\caption{Voice recognition}
\label{fig:voice_estimate_rtt}
\end{subfigure} \\
%\vspace{0.1in}
\begin{subfigure}[b]{0.6\linewidth}
	\includegraphics[width=\textwidth]{web_finish_time_example}
\caption{Web search}
\label{fig:web_finish_time_example}
\end{subfigure}
\caption{Time-line of a voice recognition flow and a web search flow.}
\label{fig:time_line}
\end{figure}

% For a voice search session, both the voice recognition and the web search phases contribute to the performance. However, they could have very distinct behavior as seen from the server-side. This is because as shown in Figure \ref{fig:time_line}, in the voice recognition phase, the server acts as a TCP receiver, while in the web search phase, the server is the sender. Note that TCP is largely a sender-driven transmission protocol and its transmission performance is affected by the ability of the sender to handle congestion events, as well as the ability of the receiver to receive data. We examine the TCP performance factors of each phase for individual voice search sessions and their impact on the flow \emph{finish time}.

User perceived response time of a mobile voice search session is affected by the \emph{finish time} of the voice recognition flow as well as that of the web search flow. Here, the finish time of a flow measures the duration from the time when client initiates the connection till the time when the last byte of data is acknowledged. We define the finish time for the flows in the two phases separately as follows.

Voice recognition consists of mobile terminals uploading the voice data and server returning back the recognized query text as shown in Figure~\ref{fig:voice_estimate_rtt}. The following web search is initialized by mobile clients once they receive the recognized text (encapsulated in one packet) at $t^c_a$. That said, the finish time of a recognition flow measures the duration from $t^c_s$ to $t^c_a$. Unfortunately, we do not have these two timestamps at server side where we collected our datasets. Note that the timestamp option in TCP is disabled since otherwise, clients behind NATs (Network Address Translations) might not be able to build connections with the servers that with PAWS (Protect Against Wrapped Sequence number) \cite{rfc7323} enabled \cite{Wang:2011:USM:2018436.2018479}. Alternatively, we approximate the finish time with $T_s=t^s_t - t^s_s$. This finish time however includes the time consumed by servers to translate speech to text ($T_r=t^s_r - t^s_v$), which is not relevant to network performance\footnote{The time duration needed for translating voice to text is dependent on the used speech recognition technology. We refer interested readers to \cite{36463,schalkwyk2010your} for more details regarding speech recognition.} and therefore out of scope in this paper. We thus redefine the finish time of a voice recognition flow as $T_s-T_r$.

Web search starts when a mobile terminal initiates a connection to transmit the query text, and ends when the server receives ACKs for all returned search results as shown in Figure~\ref{fig:web_finish_time_example}. Note that for the web search phase, we only consider flows carrying search results that are dynamically generated by servers based on the query and ignore those corresponding to static content such as CSS/JavaScript files. This is because the performance for static content (as opposed to the dynamic search results) can be optimized easily through CDN (Content Delivery Network) caching. The finish time at the client-side is the duration between transmitting the SYN packet $t^c_s$ and receiving all web search data $t^c_a$. We use the duration $T_s=t^s_a - t^s_s$ measured at the server-side to approximate the latency that the user perceives. Again, we excluded the time consumed by servers processing the query $T_r=t^s_r - t^s_q$ and obtained the finish time as $T_s-T_r$.

\begin{figure}[t]
\centering
\begin{subfigure}[b]{0.8\linewidth}
	\includegraphics[width=\textwidth]{voice_finish_time}
\caption{Voice recognition}
\label{fig:voice_finish_time}
\end{subfigure} \\
%\vspace{0.1in}
\begin{subfigure}[b]{0.8\linewidth}
	\includegraphics[width=\textwidth]{web_finish_time}
\caption{Web search}
\label{fig:web_finish_time}
\end{subfigure}
\caption{CDFs of finish time in the two phases.}
\label{fig:finish_time}
\end{figure}


Figure~\ref{fig:finish_time} plots the cumulative distribution (CDF) of the finish time of TCP flows in the two phases, where $x$-axis is in log scale. Interestingly, we find that in both phases 3G flows experience a shorter finish time than WiFi flows. Furthermore, some flows experience a very large finish time compared to others, displaying a heavy tailed behavior that was also found in Google search \cite{flach2013reducing}. For example, while most of the voice recognition flows finish in 100ms, a non-negligible fraction of flows (5\% in WiFi, and 2\% in 3G) take more than 1 second to upload the voice data. 

%Figure~\ref{fig:web_finish_time} shows the CDF of finish time of web search flows. In the figure, the overall finish time of flows in 2G and 3G (with median values 0.013s and 0.011s) in also shorter than that of flows in WiFi network (with median value 0.2s). More than 25\% of flows in all networks experience finish time less than 0.1 second. However, 5\% of flows in 2G network and 18\% in WiFi network experience finish time more than 1 second. Furthermore, about 3\% of flows in WiFi network are with finish time more than 10 seconds, which is a great performance degradation.

A comparison of the TCP flow finish time between voice recognition and web search in Figure~\ref{fig:finish_time} shows that the finish time in web search contributes to the majority of the user-perceived (network-related) performance of a voice search session. However, it does not mean that understanding the performance of voice recognition flows is not as important as for web search flows. Indeed, voice recognition is the first step of the whole voice search process, and therefore a long recognition time would certainly result in bad user experience, potentially even early termination of the process.

% These observations motivate our analysis of the TCP performance and its impact on the flow finish time in each phase of the mobile voice search.

% some of voice recognition flows suffer from timeout retransmission, which is a great performance degradation~\cite{flach2013reducing}. Second, a non-negligible fraction of voice recognition flows are terminated before voice data transmission completes. These bad user experiences also inspire us to understand what factors and how they impact user-perceived performance in voice recognition flows.

We were motivated by the above observations to have an in-depth analysis of the TCP performance in mobile voice search. To this end, we focus on the following three most important aspects of TCP performance.

\begin{itemize}
	\item {Round Trip Time (RTT):} RTT is a commonly used indicator of TCP performance, especially for short TCP flows like search flows.
	
	\item {Number of lost/disordered packets:} Both packet loss and packet reordering could affect the transmission time. Packet loss leads to a reduction of the sender congestion window. Although packet reordering does not reduce the congestion window, it can prevent the congestion window from growing and may trigger spurious retransmissions.
	
	\item {Timeout Retransmission:} In timeout retransmission, the TCP sender has to wait for a RTO (Retransmission Timeout) before retransmitting the lost packet. The RTO can be tens of RTTs. Such kind of ``expensive'' retransmissions can degrade TCP performance, especially for short flows like those considered in this paper~\cite{flach2013reducing}.

\end{itemize}

%When examining the impact of the above factors on TCP finish time, we also consider the potential impact of TCP flow size. We are particularly interested in the disparity that might exist when issuing voice searches from 3G and WiFi. 
TCP is a sender-driven transmission protocol and its performance is mainly affected by the server-side factors. However, as shown in Figure \ref{fig:time_line}, the server acts as a TCP receiver in the voice recognition phase. As such, we cannot obtain the statistics of the above TCP performance factors of the sender in the voice recognition phase. Alternatively, we infer as much information as we can for these factors. The server in the web search phase is on the other hand the TCP sender, and thus we can have a better view of how the TCP sender behavior and its impact on the finish time. In addition, we are particular interested in the disparity that might exist when issuing voice searches from 3G and WiFi. 

%!TEX root = main.tex

\section{Analysis of Voice Recognition Flow}

\subsection{Analysis Framework}

\begin{figure}[th]
\centering
	\includegraphics[scale=0.7]{voice_flow_rtt}
\caption{Time-line in voice recognition flow.}
\label{fig:voice_flow_rtt}
\end{figure}

Figure~\ref{fig:voice_flow_rtt} shows the time-line in voice recognition flow. From client side, the finish time is from that client initiates the connection to that client receives all acknowledgments of the voice data, \ie $t^c_a - t^c_s$. From server side, finish time is approximated as $(t^s_v - t^s_s) + (t^s_t - t^s_r)$, where $t^s_v$ is the time that server receives all voice data. As mentioned above, at most 3 RTT's could be measured from server side in voice recognition, including $t^s_a - t^s_s$ and $t^s_t - t^s_r$ in the figure, as well as the RTT when server terminates connection (not shown in the figure).

RTT is ambiguous when the corresponding segment is retransmitted. To investigate the impact of RTT on finish time, we use the following strategy to determine the baseline RTT. If there is no retransmission of SYN packet, $t^s_a - t^s_s$ is used as the RTT of the flow. Otherwise, the RTT when server terminates the connection is used, if the FIN packet is not retransmitted. When both of the two RTT's above are ambiguous, $t^s_t - t^s_r$ is exploited. We have verified in the web search dataset that $t^s_a - t^s_s$ is the minimal RTT in more than 60\% of flows, and is less than 2 times of the minimal RTT in 95\% of flows. Thus it is reasonable to assume that the measured RTT from server side could represent the minimum RTT from client side in voice recognition.

\begin{figure*}[th]
\centering
	\includegraphics[scale=0.7]{voice_flow_estimate_retrans}
\caption{Server could not distinguish packet loss and reordering in voice recognition.}
\label{fig:voice_flow_estimate_retrans}
\end{figure*}

Figure~\ref{fig:voice_flow_estimate_retrans} exemplifies that server could not distinguish packet loss and reordering when receiving data from client. In Figure~\ref{fig:voice_flow_estimate_retrans}(a), there are packets reordered by network, which could not be distinguished from fast retransmission in Figure~\ref{fig:voice_flow_estimate_retrans}(b). From server side, it could not distinguish whether the packet is retransmitted or reordered by network. In Figure~\ref{fig:voice_flow_estimate_retrans}(c), client retransmitted segment 5 when retransmission timer is triggered. From server side, it could not identify packet loss, but only feels that the interval between segment 4 and segment 5 is long. Server could sense unnecessary retransmission when receiving data packet twice, and notifies the spurious retransmission to client via Duplicate SACK~\cite{rfc3078}, which is shown in Figure~\ref{fig:voice_flow_estimate_retrans}(d).

\subsection{Causal Analysis of Finish Time}

\begin{table}[th]
\centering
\renewcommand{\arraystretch}{1.2}
\caption{Percentage of abnormal flows in different ISP's.}
\label{tab:voice_stats}
\begin{tabular}{l|c|c|c}
	\toprule
	 & CT & CU & CM \\
	\midrule
	packet reordering & 2.3\% & 3.1\% & 4.9\% \\
	\hline
	SYN retransmission & 2.1\% & 1.7\% & 4.5\% \\
	\hline
	timeout retransmission & 5.3\% & 4.9\% & 4.9\% \\
	\hline
	incomplete transmission & 0.2\% & 0.3\% & 2.4\% \\
	\bottomrule
\end{tabular}
\end{table}

Figure~\ref{tab:voice_stats} shows the percentage of abnormal flows in each ISP. A flow is abnormal if it encounters packet loss, reordering, timeout retransmission, or incomplete retransmission. In the table, about 2$\sim$5 percentage of flows are with packet reordering. About 5\% of flows experience timeout retransmission, which is larger than that of flows with packet reordering. The reason is as follows. First, 

Even though the flows are short, with no more than 6 data packets, there are flows 

\begin{table}[th]
\centering
\renewcommand{\arraystretch}{1.2}
\begin{tabular}{l|l|l|l|l|l|l}
	\toprule
	& \multicolumn{3}{c|}{ RTT(s) } & \multicolumn{3}{c}{ reordering (\#(pkts))} \\
	\midrule
	ISP & CT & CU & CM & CT & CU & CM \\
	\midrule
	wifi & 0.1 & 0.088 & 0.198 & 0.026 & 0.034 & 0.056 \\
	\hline
	2G & 0.062 & 0.029 & 0.07 & 0.079 & 0.046 & 0.057 \\
	\hline
	3G & 0.039 & 0.025 & 0.052 & 0.078 & 0.038 & 0.069 \\
	\hline
	4G & - & - & 0.05 & - & - & 0.01 \\
	\bottomrule
\end{tabular}
\caption{The average RTT and number of disordered packets for each access type.}
\label{tab:voice_access_type_stats}
\end{table}


\begin{table}[th]
\centering
\renewcommand{\arraystretch}{1.2}
\begin{tabular}{l|m{.35in}|m{.35in}|m{.35in}}
	\toprule
	& CT & CU & CM \\
	\midrule
	$[$ 0.04, 0.08 ) & 1.50 & 1.60 & 1.52 \\
	\hline
	$[$ 0.08, 0.12 ) & 1.83 & 2.69 & 1.97 \\
	\hline
	$[$ 0.12, 0.16 ) & 2.76 & 3.22 & 3.23 \\
	\hline
	$[$ 0.16, 0.20 ) & 3.72 & 5.40 & 8.06 \\
	\hline
	$[$ 0.20, $\infty$ ) & 13.2 & 25.6 & 29.2 \\
	\bottomrule
\end{tabular}
\caption{}
\label{tab:}
\end{table}

\begin{table}[th]
\centering
\renewcommand{\arraystretch}{1.2}
\begin{tabular}{l|c|c|c}
	\toprule
	ISP & CT & CU & CM \\
	\midrule
	PCC & 0.0002 & 0.002 & -0.003 \\
	\bottomrule
\end{tabular}
\caption{The Pearson Correlation Coefficients between $finish\_time$ and flow size.}
\label{tab:fin_vs_pkt}
\end{table}

In this section, we investigate how the factor(s) impact the performance in voice recognition. First, we use PCC to inspect the relationship between $finish\_time$ and flow size in the three ISP's. The correlation coefficients are shown in Table~\ref{tab:fin_vs_pkt}. From the table, the $finish\_time$ of flows is completely irrelevant to the flow size. The reason is as follows. Most of flows contain no more than 6 data packets, which could be packed in the initial congestion window from client side.

% TODO explain the reason why flows in CU experience such a high transmission time.
\begin{figure}[!htbp]
\centering
	\includegraphics[width=3in]{voice_rtt}
\caption{The finish time under various RTT's in voice recognition.}
\label{fig:voice_rtt}
\end{figure}

Next, we study the impact of RTT on the $finish\_time$ in voice recognition. The RTT's are measured according to the strategy in Section~\ref{sec:dataset}. We group the flows into different bins by their RTT's, with 0.04 second intervals. Figure~\ref{fig:voice_rtt} plots the average $finish\_time$ of flows in each bin. From the figure, as the RTT value becomes larger, the finish time increases correspondingly. When the RTT value is less than 0.20 second, finish time has a roughly linear relationship with RTT. In fact, the correlation coefficients between the mean value of finish time in each bin and the index are more than 0.83, 0.81, 0.85 in the three ISP's. In the figure, the ratio of average finish time to the RTT value falls in the interval $[2, 4]$, except that when RTT value is larger than 0.2 second. When the RTT value is larger than 0.2 second (corresponding to 8\% of flows), the average finish time is more than 1.2 second. The reason is as follows. Larger RTT indicates that the packets are buffered in the network for longer time, which is also a signal of network congestion, like in TCP Vegas\cite{brakmo1995tcp}, FastTCP\cite{wei2006fast}. Thus when flow experiences large RTT, it is likely that the flow is traversing congested network, and may encounter packet loss, thus has exceptional large finish time.

\begin{figure}[th]
\centering
\includegraphics[width=3in]{voice_reorder}
\caption{The finish time under different number of disordered packets in voice recognition.}
\label{fig:voice_reorder}
\end{figure}

In the above, we have analyzed that from server side, packet loss and reordering could not be distinguished. In the following, we will take packet reordering as an indication of network congestion and study the impact of network congestion on finish time. In packet reordering, the number of disordered packets is defined as follows. In receiving sequence $S_2, S_4, S_5, S_3$, the packet $S_3$ is disordered, thus the number of disordered packets is 1. Figure~\ref{fig:voice_reorder} shows the average finish time under different number of disordered packets. From the figure, in all three ISP's, when there are disordered packets, flows experience significant performance degradation.

Even when there is only 1 disordered packet, the average finish time is 2$\sim$8 times higher than that of flows without disordered packet. This could be explained as follows. If server receives packets which are not successive, it feeds back to the client with SACK. When receiving SACK, client does not retransmit the packet in the hole (\ie the disordered packet) immediately, until client receives 3 SACK's or ACK of that packet, or retransmission timer is triggered. If the packet is dropped, client needs to wait at least one RTT to retransmit the packet. If there are not sufficient number of SACK's, client has to rely on timeout retransmission (RTO). The RTO timer, as an estimate of the RTT and variation in RTT, is usually highly conservative, which is set to tens of, or even hundreds of RTT. Especially in CU, when flows have more than 2 disordered packets, the average finish time is 20 times higher than that of flows without packet reordering.

The significant impact of network congestion on finish time motivates us to investigate the impact of timeout retransmission on finish time in voice recognition. We use the following trick to identify each RTO. For each packet, there is an estimated arrival time, calculated according to the time of preceding and subsequent packets. If the gap between actual arrival time and the estimated arrival time is larger than $max(200ms, 3 RTT)$, it is identified as a RTO.

As there are no more than 6 data packets in most of voice recognition flows, if there are 3 or more packets dropped, client has to rely on RTO to recover the transmission. 

% Figure~\ref{fig:voice_rto} 

\subsection{Summary of Voice Recognition Analysis}
%!TEX root = main.tex

\section{Analysis of Web Search}
\label{sec:web_search}

\begin{figure}[th]
\centering
	\includegraphics[width=0.5\linewidth]{web_finish_time_example}
\caption{Time-line in web search flow.}
\label{fig:web_finish_time_example}
\end{figure}

Web search starts when mobile terminal initiates connection to transmit query text, and ends when server receives ACKs for all returned search results. Figure~\ref{fig:web_finish_time_example} shows the typical time-line in web search flows. The finish time at client side is the duration from transmitting SYN packet $t^c_s$ to receiving all web search data $t^c_a$. We use the duration $T_s=t^s_a - t^s_s$ measured at server side to approximate the latency that user perceives. Again, we excluded the time consumed by servers processing the query $T_r=t^s_r - t^s_q$ and obtained the finish time as $T_s-T_r$.

For each data packet which is not retransmitted, the measured RTT is the duration from that server transmits the packet, to that server receives the acknowledgment for it. In web search progress, we record as many RTT's as possible. When each new RTT is measured, the RTO value is updated by $RTO = SRTT + max(200ms, 4 RTTVAR)$, where $SRTT$ and $RTTVAR$ mean smoothed RTT and RTT variation respectively. For a retransmitted packet, if the duration between its retransmission time and the time that last packet is transmitted is larger than the calculated RTO, we could infer that it is a timeout retransmission. For each retransmitted packet, if server would not receive D-SACK, such a retransmission recovers a real packet loss, otherwise, the packet is not dropped and the retransmission is spurious. Note that a data segment could be retransmitted twice or more if the previously retransmitted packet is regarded as lost.

In the above, we have seen that finish time in voice recognition is strongly related to the RTT value. Here we investigate the impact of RTT on finish time of web search flows. The distribution of RTT in web search is similar to that in voice recognition, which is not shown due to space limit. We use Kendall correlation to determine their relationship. The coefficient between RTT and finish time is 0.15, showing weak relationship between them. The reason is as follows. Web search result usually contains more than 10 data packets (shown in Figure~\ref{fig:web_flow_size}), which could be transmitted in 1 RTT. Moreover, how many data packets could be transmitted in one RTT is determined by congestion window size, which varies when receiving acknowledgment and packet loss event. Flows have different congestion window sizes due to different amount of congestion events in the paths they traverse. Thus RTT has limited impact on finish time in web search. 

\begin{figure}[th]
\centering
	\includegraphics[width=0.8\linewidth]{web_flow_size}
\caption{The distribution of flow sizes in web search.}
\label{fig:web_flow_size}
\end{figure}

Figure~\ref{fig:web_flow_size} plots the distribution of flow size (measured by the number of web search result packets) in web search. The flow size varies from 1 packet to 100 packets with a median around 10 packets. Such a small flow size implies that any TCP performance degradation (like a high packet loss) could exert a large impact on the user perceived latency (i.e. finish time) \cite{flach2013reducing}. We also observe a smaller flow size of WiFi search than that of 3G search, which might be due to the difference in search behavior \cite{Song:2013:EEU:2488388.2488493}. We use Kendall correlation to determine whether flow size is relevant to finish time. The coefficient is 0.018, which demonstrates their irrelevance.

As a TCP sender in web search, the server can measure abundant TCP performance factors and behavior, enabling us to perform a detailed analysis of the TCP performance and its impact on the finish time. In what follows, we first characterize the finish time distribution in 3 TCP stages and then examine the impact of TCP performance factors.

\subsection{TCP Stage Analysis}

\begin{figure}[th]
\centering
\includegraphics[width=0.8\linewidth]{web_three_stages}
\caption{The three stages in web search flows.}
\label{fig:web_three_stages}
\end{figure}

A TCP comprises 3 stages: \emph{3-way handshake}, \emph{slow start}, \emph{congestion avoidance}, which is exemplified in Figure~\ref{fig:web_three_stages} using a web search flow from our dataset, where $y$-axis shows the TCP sequence number. The TCP 3-way handshake (3WHS) stage ideally (i.e. without packet loss, delay and reordering) completes within 1 RTT. The server then enters the slow start stage, during which server does not encounter any packet loss or reordering event, and thus enlarges the congestion window by 1 segment size for each received ACK. The server enters congestion avoidance stage once it estimates a packet loss\footnote{The packet can actually be delayed or lost.}. In this stage, the server reduces the congestion window when detecting packet loss through fast retransmit\cite{jacobson1988congestion} and compels the congestion window to grow from 1 segment size when detecting packet loss through RTO. Note that the congestion avoidance stage here starts when server detects congestion event and ends till the flow finishes, which is different the TCP congestion avoidance state in TCP/IP stack.

%Figure~\ref{fig:web_three_stages} gives an exemplified flow with 3 TCP stages that we define. The $y$-axis shows the TCP sequence number. In the figure, server takes 1.8s to establish connection, 3.1s to transmit 35 data packets in slow start stage, and 15.3s to transmit the left 48 data packets in congestion avoidance stage. In the following, we use the criteria of partitioning to break the analysis down into the three stages that flows experience.

\subsubsection{3-way Handshake}

\begin{figure}[th]
\centering
\includegraphics[width=0.8\linewidth]{web_handshake_time_ratio}
\caption{The ratio of time in 3-way handshake to finish time in each flow.}
\label{fig:web_handshake_ratio}
\end{figure}

We first examine how much of the time spent in the 3WSH stage in Figure~\ref{fig:web_handshake_ratio}, where we plot the ratio of the time in 3WSH to the finish time. 3G and WiFi show similar ratio of time spent in this stage. We observe that the time for connection establishment can take up to half of the finish time for 30\% flows. More surprisingly, about 8\% of the WiFi flows and 3\% of 3G flows spend 70\% of their time during this stage. The small flow size is one of the reason for this observation. The maximum flow size is around 100 packets as shown in Figure \ref{fig:web_flow_size}, implying the data can be transmitted within only a few RTTs if no congestion event happens. such a short period of transmission time leads to a relatively large portion of time for connection establishment.


%From the figure, flows in cellular network have similar ratio of time in 3-way handshake to that in WiFi network (with median value 0.3). If the 3WSH could be removed from finish time, the user-perceived web search latency would be reduced by 30\% in more than half of the flows. Moreover, there are 8\% of flows in WiFi network consuming 70\% of their time in 3WSH.

%The unexpectedly high ratio of 3WSH could be introduced by two reasons. First, most of web search flows contains packets ranging from 1 to 100, these data could transmitted in 1 to 4 RTT's if there is no congestion event. Thus 3WHS, without transmitting any data, occupies a large fraction of time in short flows. Second, there are a non-negligible fraction of flows experiencing SYN retransmission during 3WHS stage. 

Another important reason that explains the unexpectedly the long time in 3WSH stage (i.e. $>70\%$ of finish time) is the timeout retransmissions in this stage. As the data cannot be transmitted before a TCP connection is established, a loss of SYN will lead to a timeout retransmission, which takes 1 second (\ie the initial RTO) to retransmit the SYN.  We find that 4.4\% of WiFi search flows and 0.5\% of 3G flows experience at least one SYN timeout retransmission. We envision a possible way to mitigate the costly 3WSH in such short flows where clients maintain long-term TCP connections to the web search servers.

\subsubsection{Slow Start Stage}

\begin{figure}[th]
\centering
\includegraphics[width=0.8\linewidth]{web_slowstart_time_ratio}
\caption{The ratio of time in slow start to the time in data transmission.}
\label{fig:web_ss_time_ratio}
\end{figure}

The slow start stage and the congestion avoidance stage constitute the data transmission period. During the slow start stage, the data transmission throughput monotonically increases as the congestion window is doubled after each RTT. Intuitively, a longer time in slow start stage, the shorter time to complete the data transmission. Figure~\ref{fig:web_ss_time_ratio} shows the ratio of time in slow start stage to the time spent in data transmission (\ie the sum of time in slow start and congestion avoidance stages). Note that the $y$-axis is capped at 0.1. Regardless of the access type, more than 90\% of the flows can be finished in the slow start stage. In other words, about 10\% of the flows have to experience the congestion avoidance stage, which could lead to an increased finish time as we will see in the following analysis.

Another interesting observation is that 1.5\% of the WiFi flows is unable to transmit any data in the slow start stage as the first packet is dropped. We indeed find of the flows that experience the congestion avoidance stage, the packet loss happens within the first congestion window (i.e. within the first 10 packets) for 80\% of WiFi flows, while this percentage is only 20\% for 3G flows, implying a reconsideration of the initial congestion window configuration.

\subsubsection{Congestion Avoidance Stage}

\begin{figure}[th]
\centering
\includegraphics[width=0.8\linewidth]{web_ca_prac_over_est}
\caption{CDF of the extra time introduced by the congestion avoidance stage.}
\label{fig:web_ca_round}
\end{figure}

We then examine how much extra time the congestion avoidance stage introduces for individual flows. To this end, we first compute an estimated ideal transmission time ($T_e$) without this stage for each flow. The estimated ideal time is $T_e = \frac{\#(ca\_pkts)}{cwnd} \times RTT$, where $\#(ca\_pkts)$ is the number of packets transmitted in this stage, $cwnd$ is the congestion window size at the server at the time when entering this stage. However, we cannot obtain $cwnd_e$ from the traces and thus alternatively use the number of in-flight packets at the time when entering the stage to approximate $cwnd_e$ \cite{rfc56812009tcp}. Then we compute the ratio of the time spent in this stage obtained from the trace (denoted as $T_r$) to the estimated idea transmission time. We use this ratio to measure the extra time introduced by the congestion avoidance stage and show the results in Figure \ref{fig:web_ca_round} for those flows experiencing this stage.

We observe a surprisingly high ratio for WiFi flows. The median ratio is as high as 3.2 and 20\% of the WiFi flows has a ratio more than 11, meaning a 10 times extra time is introduced by this stage. The high ratio can be attributed to two factors. First, the real congestion window of these flows is much smaller than $cwnd_e$ due to packet losses. Second, some lost packets require the timeout retransmission to recover, in which cases server must wait for a RTO without data transmission. It is interesting to see that 3G flows have a ratio no more than 2 for 95\% of the flows, meaning that 3G flows have a low packet loss rate and experience fewer timeout retransmissions. In what follows, we will examine in detail the impact of packet loss and timeout retransmission on finish time. 

\subsection{Impact of packet loss}
\label{sec:web_pkt_loss}

% \begin{figure}[th]
% \centering
% \includegraphics[width=0.8\linewidth]{web_loss_fintime_cdf}
% \caption{The distribution of finish time with and without packet loss in web search.}
% \label{fig:web_loss_fintime_cdf}
% \end{figure}

As we have analyzed above, packet loss would degrade TCP performance by both reducing congestion window and possibly triggering timeout retransmission. The the web search dataset, about 9\% of flows in both WiFi and 3G network experience packet loss. However, the packet loss rates in WiFi and 3G network are 3\% and 0.9\% respectively.

% Figure~\ref{fig:web_loss_fintime_cdf} shows the distribution of finish time with and without packet loss in web search. In the figure, there are distinct performance gap between flows with and without packet loss. The mean finish time in flows without packet loss is about 2.5 - 5 times smaller than that of flows with packet loss.

\begin{figure}[th]
\centering
\includegraphics[width=0.8\linewidth]{web_loss_finish_time}
\caption{Finish time under different number of lost packets.}
\label{fig:web_loss_finish_time}
\end{figure}

To quantitatively determine the impact of packet loss, we investigate how much packet loss events could increase finish time. Figure~\ref{fig:web_loss_finish_time} shows the finish time of web search flows under different number of lost packets. In the figure, flows are grouped by counting how many lost packets in each flow, and the average finish time, as well as 5 and 95 percentile of finish time in each group are calculated. When there is no packet loss, the average finish time of flows in WiFi network is about 1 second, while that of flows in 3G network is only 0.3 second, which is induced by smaller RTT in 3G network. When the number of lost packets is 1 or 2, the finish time in WiFi network increases to about 2.7 second, while that of flows in 3G network hardly increases, compared with those without packet loss. When the number of lost packets increases to 3 or more, the finish time in WiFi increases to 8.1 second, while that in 3G network increases to 0.51 second. Thus, compared with flows in 3G network, flows in WiFi experience much longer finish time under the same level of packet loss, and will suffer more severe performance degradation if the number of lost packets increases. 

\subsection{Impact of timeout retransmission}

\begin{figure}[th]
\centering
\includegraphics[width=\linewidth]{web_loss_rto_ratio}
\caption{The fraction of flows with timeout retransmission under different number of lost packets.}
\label{fig:web_loss_rto_ratio}
\end{figure}

To understand how each packet loss impact finish time, we distinguish timeout retransmission from fast retransmit for each retransmission of lost packet. After that, we count how many timeout retransmissions in the flows within each group classified above, which is shown in Figure~\ref{fig:web_loss_rto_ratio}. Note that the two sub-figures have different $y$-axis labels. From the figure, when the number of lost packets is 1, the fraction of flows with timeout retransmission in WiFi network is 16\%, much larger than that value (2\%) in 3G network. This could explain why the finish time of flows in WiFi network increases a lot, while that of flows in 3G network does nearly not increase, under 1 lost packet. When the number of lost packets increases, the fraction of flows with timeout retransmission increases in both networks. Combined with Figure~\ref{fig:web_loss_finish_time}, we could conclude that the finish time under packet loss in each network is almost proportional to the amount of timeout retransmissions there are. This indicates that when there is packet loss, timeout retransmission dominates the finish time in web search flows.

The different amount of timeout retransmission in the two networks under the same number of lost packets motivates us to investigate how timeout retransmission is triggered. Essentially, there are two scenarios in which sender has to appeal timeout retransmission for loss recovery. The first scenario is when a transmitted packet is dropped by network, sender has to wait for RTO for recovery (noted as \emph{double retransmission}). 

The second scenario is that sender could not receive sufficient number of duplicate acknowledgments (dupacks), which is the prerequisite of fast retransmit. The insufficiency of dupacks could be induced by various reasons. When packet loss happens in the last 3 packets of a flow, sender could not receives 3 dupacks. Timeout retransmission is triggered due to insufficient number of data packets, which is named \emph{tail retransmission}~\cite{flach2013reducing}. When the congestion window is small (\eg 1 segment size after timeout retransmission), packet loss has to be recovered by timeout transmission, which is named \emph{small congestion window retransmission}. Even if there are sufficient data packets for transmission, and congestion window size is large enough, packet loss may also be recovered by timeout retransmission. Considering the data packets $s_1, s_2, \cdots, s_n$ ($n > 3$), in which $s_1$ is dropped, the acknowledgments for $s_2, \cdots, s_n$ reach the sender with a long latency, after timeout transmission occurs. This could be induced by severe network latency jitters, or by misbehaving middle-boxes blocking the dupacks~\cite{honda2011isit}, which is named \emph{packet delay retransmission}.

We propose a simple strategy to determine how each RTO is triggered in Algorithm~\ref{alg:rto}. In the algorithm, there are other timeout retransmissions which could not be classified as one of the reasons above, such as when all packets in the windows are dropped. 

\begin{algorithm}
	\caption{Process of determining the cause of RTO.}
	\label{alg:rto}
	\begin{algorithmic}[1]
		\Procedure{ParseRTO}{$RTO$}
			\If {packet has been retransmitted}
				\State \textbf{return} $double\_retransmission$
			\ElsIf {position to tail $\le$ 3}
				\State \textbf{return} $tail\_retransmission$
			\ElsIf {\#(in-flight packets) = 1}
				\State \textbf{return} $small\_cwnd\_retransmission$
			\ElsIf {only 1 in-flight packet is dropped}
				\State \textbf{return} $packet\_delay\_retransmission$
			\Else
				\State \textbf{return} $others$
			\EndIf
		\EndProcedure
	\end{algorithmic}
\end{algorithm}

\begin{table}[th]
\caption{The ratio of timeout retransmissions in each type.}
\label{tab:rto_type}
\centering
\renewcommand{\arraystretch}{1.0}
\begin{tabular}{c|C{1.1cm}|C{1.1cm}}
	\hline
	\textbf{timeout retx type} & WiFi & 3G \\
	\hline
	tail retx & 38.3\% & 69.6\% \\
	\hline
	double retx & 33.8\% & 13.5\% \\
	\hline
	packet delay retx & 27.4\% & 16.1\% \\
	\hline
	small cwnd retrx & 0.3\% & 0.6\% \\
	\hline
\end{tabular}
\end{table}

According to the algorithm, we determine how many timeout retransmissions belong to each type. Table~\ref{tab:rto_type} shows the fraction of each type. From the table, tail retransmission occupies the largest fraction in both networks, which illustrates the severity of tail packet loss in short flows~\cite{flach2013reducing}. Moreover, compared with those in 3G networks, flows in WiFi network suffer more from double retransmission (33.8\%) as well as packet delay retransmission (27.4\%). The former occurs in flows with two or more lost packets, while the latter could occur even when there is only one packet loss. This explains why there are so many timeout retransmissions in flows in WiFi network. As the three major kinds of timeout retransmission are induced by various reasons: flow property (tail retransmission), TCP mechanism (double retransmission), and misbehaving middle-boxes (packet latency retransmission), to eliminate timeout retransmission may require a combination of efforts from many sides, like improving TCP protocol, application-level optimization, as well as misbehavior troubleshooting in middle-boxes.

\subsection{Summary of web search analysis}

The key observations on web search flows are summarized below.

\begin{itemize}
	\item Compared to the impact of RTT on voice recognition flow, RTT has limited affect on the transmission time per packet in web search.
	\item More than 50\% of flows spend 30\% of their total time on establishing connections.
	\item Flows in WiFi network spend tens of times more than estimated time in congestion avoidance due to timeout retransmission.
	\item Most of timeout retransmissions in both networks are induced by packet loss at flow tail. However, about 33.8\% of timeout retransmissions in WiFi are induced by loss of retransmitted packet.
\end{itemize}

%!TEX root = main.tex

\section{Summary and Discussion}
\label{sec:discuss}

Our study mainly focuses on the TCP performance of voice search service at server side. As voice search session consists of voice recognition and web search phases, we inspect the main factors that may impact the finish time of flows corresponding to each phase. After quantitative measurement of the flows, we obtain the following findings.

\begin{itemize}
	\item Compared to flows in 3G network, WiFi flows experience longer finish time in both phases. The deficiency of WiFi flows are induced by both longer RTT and higher packet loss rate.
	\item If timeout retransmission occurs in a flow, it would dominate the finish time. As flows in both phases are short (less than 100 packets), which only takes less than 5  RTT's for transmission if there is no packet loss. While, timeout retransmission may occupy tens of, or even hundreds of RTT's for recovery.
	\item Timeout retransmission could be triggered by various reasons, like SYN retransmission, tail retransmission, double retransmission, and packet delay retransmission. The reasons are located in various entities, like flow property, TCP mechanism, and misbehaving middleboxes.
	\item Tail retransmission occupies the largest fraction in both WiFi and 3G flows, due to the small size of flows. In WiFi flows, higher packet loss rate also induces double retransmission, which involves about one third of all timeout retransmissions.
	\item SYN retransmission occurs in a non-negligible fraction of WiFi flows. SYN packet loss takes 1 second for recovery and also compels the congestion window to start from 1 segment size.
\end{itemize}

The above findings could better guide service providers to optimize the transmission performance in voice search service. First, as smaller RTT leads to shorter finish time, service provider could deploy front-end servers nearer to clients to achieve shorter latency. Second, mobile applications could maintain TCP connections before acquiring services, which will mitigate the impact of SYN retransmission on user-perceived performance. Third, the 3G network access and WiFi network access could be combined under multipath TCP for more reliable service access~\cite{UM-CS-2012-022}, which enables dynamically offloading traffic from congested network, without breaking existing connections. Last but not least, service provider could deploy solutions like TLP~\cite{flach2013reducing} to reduce the number of tail retransmission in short flows. However, other types of timeout retransmission require great effort from service provider for mitigation, such as redesigning TCP retransmission mechanism to eliminate double retransmission, and cooperating with network provider to debug misbehaving middleboxes.

The limitation of our data collection also restricts our understanding of the impact of TCP performance in voice search service. First, our dataset are collected at server sides, thus we could not determine the exact packet loss information in voice recognition flow. As packet loss plays an important role in TCP performance, which also demonstrates the difficulty of performance analysis of TCP uploading flows at server side. Second, as the flows belonging to each phase are collected separately, we could not match each \emph{voice recognition}-\emph{web search} pair. Thus we are not capable of quantitatively understanding how each phase impacts the user-perceived performance in voice search, and how each network factor impact different in the two phases. Third, from our dataset, we find that RTT in WiFi network is about 2-3 times larger than that in 3G network, which is contradictory to previous studies like \cite{sommers2012cell}. This difference might be induced by the unique network characteristics in the region network where we collected the dataset. However, we could verify our findings through more data collection from multiple region networks.

We believe that we could obtain more findings if the limitation of dataset are solved, which are left as future work.
%!TEX root = main.tex

\section{Related Work}
\label{sec:related}

There is a large and growing body of work~\cite{chen2012network,deshpande2010performance,sommers2012cell,sharma2010goodput,yu2014can} characterizing the performance of mobile services in WiFi and cellular networks. Studies that are most closely related to ours have focused on analyzing how each network factor impacts the user-perceived performance in mobile services, like throughput and latency. Chen \etal~\cite{chen2012network} measured the network performance of mobile terminals in campus WiFi networks. They identified the dominating factors that affect network performance, and found that short flows in mobile terminals benefit from the large initial congestion window. Deshpande \etal~\cite{deshpande2010performance} measured cellular and WiFi performance in a metro area, via a laptop driven through the targeted area. Their results showed the higher availability yet lower performance of cellular networks, versus lower availability yet higher performance of WiFi networks. Sommers \etal~\cite{sommers2012cell} obtained similar results to \cite{deshpande2010performance}, through measurements with crowd-sourced data. Yu \etal~\cite{yu2014can} measured several popular mobile video telephony applications, and found that mobile video quality is highly vulnerable to bursty packet loss and high latency. 

There is also a growing literature~\cite{nikravesh2014mobile,deng2014wifi,huang2013depth,UM-CS-2012-022} on characterizing mobile network performance and usage. Nikravesh \etal~\cite{nikravesh2014mobile} inspected the cellular network performance, and found that the performance variations are partially induced by different carriers as well as spatial and temporal patterns. Huang \etal~\cite{huang2013depth} studied the impact of network protocol and application behaviors on performance in LTE. They found that TCP connections in LTE significantly under-utilize the available bandwidth, which is induced by both application behavior and TCP parameter setting. Chen \etal~\cite{UM-CS-2012-022} characterized the 3G/4G networks of US network carriers, in terms of throughput, packet loss, and latency. Moreover, they pointed out the benefits of multipath TCP when crossing congested network, for reliable and efficient TCP transfer.

The high packet loss rate and misbehavior of middleboxes in mobile networks may trigger costly timeout retransmission, leading to performance degradation for short flows. Jiang \etal~\cite{jiang2012tackling} studied bufferbloat and its impact on TCP performance in cellular networks. They revealed the severity of the bufferbloat problem and the induced long latency in current cellular networks. Wang \etal~\cite{Wang:2011:USM:2018436.2018479} studied the carriers' middlebox policies by conducting active measurement. They found that some middleboxes buffer disordered packets, which prevents the sender from receiving enough dupacks, causing timeout retransmissions. Flach \etal~\cite{flach2013reducing} evaluated the impact of packet loss in the flow tail on the performance in short flows. The results showed that packet loss at flow tail tend to trigger timeout retransmissions due to insufficient new data for transmission.

Compared to previous works, to the best of our knowledge, our paper is among the first to systematically analyze the impact of mobile network on the performance of both TCP upload and download in WiFi and cellular networks. We have quantitatively studied how each network factor, such as RTT, packet loss, reordering, timeout retransmission, impacts the TCP performance. We also have analyzed how each costly timeout retransmission is triggered and discussed possible solutions.

%!TEX root = main.tex

\section{Conclusion}
\label{sec:conclude}

%long version
A mobile voice search session consists of a first voice recognition phase followed by a web search phase. By gathering a unique dataset collected from one of the top search providers in China, we analyse the traffic from the voice search service based on packet-level traces. We find a considerable fraction of the flows suffering from very long completion time in both phases of the service, which motivates us to perform a detailed analysis on the TCP performance and its causes. We focus on the disparity that might exist between TCP upload (voice search) and download (web search result) and also the possible difference when using 3G and WiFi connection. We observed a better and more consistent performance of 3G flows compared to WiFi flows, mostly due to higher packet loss rate and larger RTT in WiFi. In fact, WiFi flows are more likely to use timeout retransmission for packet loss recovery. Timeout retransmission, if it ever occurs, dominates the flow completion time. RTT on the other hand plays an important role under no packet loss. We also classified the timeout retransmissions and found the different compositions of timeout retransmissions between WiFi flows and 3G flows. Overall, our observations constitute valuable insights for improving the user-perceived performance in voice search service, as well as shed light on the intricacies of TCP for short flows.

%short version
% A mobile voice search session consists of a first voice recognition phase followed by a web search phase. Analyzing the TCP performance of mobile voice search at server side can shed useful lights on the understanding of TCP short flows for both upload and download. By observing a unique dataset collected from a voice search service, we find a considerable fraction of outlier flows that suffer from very long completion time in both phases, which motivates us to perform a detailed analysis on the TCP performance and its causes. We focus on the disparity that might exist between TCP upload and download flows and also the possible difference when searching using 3G and WiFi connection. In particular, we analyzed timeout retransmission, an expensive operation for packet loss recovery as it dominates the completion time once it happens. We have made several key observations, which constitute valuable insights for improving the user-perceived performance in voice search service.


\bibliographystyle{ieeetr}
\bibliography{reference}

\end{document}
